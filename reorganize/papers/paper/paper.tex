\documentclass[9pt,twocolumn,twoside,lineno]{pnas-new}
% Use the lineno option to display guide line numbers if required.
% Note that the use of elements such as single-column equations
% may affect the guide line number alignment. 

\templatetype{pnasresearcharticle} % Choose template 
% {pnasresearcharticle} = Template for a two-column research article
% {pnasmathematics} = Template for a one-column mathematics article
% {pnasinvited} = Template for a PNAS invited submission

\title{Dynein walks like an Imperial AT-ST}

% Use letters for affiliations, numbers to show equal authorship (if applicable) and to indicate the corresponding author
\author[a,c,1]{Author One}

\affil[a]{Oregon State University}

% Please give the surname of the lead author for the running footer
\leadauthor{Lead author last name} 

% Please include corresponding author, author contribution and author declaration information
\authorcontributions{Please provide details of author contributions here.}
\authordeclaration{Please declare any conflict of interest here.}
\equalauthors{\textsuperscript{1}A.O.(Author One) and A.T. (Author Two) contributed equally to this work (remove if not applicable).}
\correspondingauthor{\textsuperscript{2}To whom correspondence should be addressed. E-mail: author.two\@email.com}

% Keywords are not mandatory, but authors are strongly encouraged to provide them. If provided, please include two to five keywords, separated by the pipe symbol, e.g:
\keywords{dynein $|$ Brownian dynamics $|$ powerstroke $|$ modeling} 

\begin{abstract}
Please provide an abstract of no more than 250 words in a single paragraph. Abstracts should explain to the general reader the major contributions of the article. References in the abstract must be cited in full within the abstract itself and cited in the text.
\end{abstract}

\dates{This manuscript was compiled on \today}
\doi{\url{www.pnas.org/cgi/doi/10.1073/pnas.XXXXXXXXXX}}

\begin{document}

% Optional adjustment to line up main text (after abstract) of first page with line numbers, when using both lineno and twocolumn options.
% You should only change this length when you've finalised the article contents.
\verticaladjustment{-2pt}

\maketitle
\thispagestyle{firststyle}
\ifthenelse{\boolean{shortarticle}}{\ifthenelse{\boolean{singlecolumn}}{\abscontentformatted}{\abscontent}}{}

\dropcap{D}ynein is a motor protein used to generate directed force in cells. The protein is a homodimer which binds to cellular filaments known as microtubules (MTs). Each monomer has several ATPase domains arranged in a larger globular domain known as the head. This head is the site which hydrolyzes ATP and undergoes the conformational changes responsible for dynein's step. The head is attached via a long chain to the microtubule binding domain (MTBD). Dynein is an interesting structure in that it manages to coordinate ATPase chemistry at its head with MT-releasing chemistry at its MTBD, some 20nm away \cite{mt-atp-coupling}. The head has a long tail domain coming off it, which eventually dimerizes to the other monomer.\\

Dynein is unique in that it has a widely varied step size. Dynein's average step is 8 \textit{nm} in the forwards direction, but it is capable of taking 32 \textit{nm} steps in the forwards and reverse directions \cite{weihongpaper} \cite{yildizpaper}. This stochastic, varied stepping is contrasted with the much more regular 8 \textit{nm} step size of kinesin, another bipedal motor protein \cite{kinesin-step-size}. It has been suggested that dynein's long separation between its head


Dynein is known to change stepping behavior and direction as a response to forwards or backwards load \cite{responsetoload}.


\subsubsection*{SI Figures}

Lorem ipsum dolor sit amet, consectetur adipiscing elit. Mauris in aliquet quam, a cursus orci. Fusce vitae faucibus tortor, vel molestie nibh. Proin condimentum cursus mollis. Praesent euismod ligula sit amet libero elementum, sed commodo velit suscipit. Integer sed nisl vitae nibh dapibus sagittis. Vestibulum ante ipsum primis in faucibus orci luctus et ultrices posuere cubilia Curae; Nullam et semper sem. Nulla fringilla lacus quis justo tincidunt venenatis. Vestibulum aliquet ante diam, nec accumsan augue rutrum id. Nam sollicitudin magna vitae mauris pellentesque, et rhoncus felis accumsan.\\

Etiam ullamcorper elit a interdum molestie. Phasellus aliquam nulla nec orci sodales, sit amet bibendum odio pulvinar. Nullam at pulvinar lectus. Nulla facilisi. Phasellus ut magna ut nunc vehicula posuere non ac ligula. Nulla facilisi. Duis ultrices ornare turpis, eget tempus mauris pulvinar interdum.\\

Curabitur eget nisi tellus. Nullam bibendum et dolor et rutrum. Sed tincidunt quam nec laoreet tincidunt. Etiam quam erat, tempus in posuere eget, pellentesque vel ex. Nunc non turpis sagittis, condimentum ante porttitor, pulvinar risus. Proin sodales, odio vel efficitur porta, nunc ligula ornare leo, ut lobortis elit ante bibendum mauris. Sed molestie nisl at elit efficitur interdum.\\

Nam ultricies efficitur nunc, in tristique tellus auctor vitae. Aliquam ante leo, varius a ante sit amet, pretium efficitur elit. Nulla suscipit, nunc eget mollis blandit, nulla lorem venenatis ipsum, lobortis ultricies leo justo eu diam. Proin fringilla, purus non hendrerit vulputate, eros est ultrices massa, vel faucibus mi risus ac magna. Ut varius justo maximus finibus aliquam. Nullam rutrum vestibulum eros sed accumsan. Vestibulum a porttitor urna, porttitor tempus sem. Quisque dapibus aliquam pulvinar. Fusce vehicula nisl a tellus semper vulputate.\\

Nunc lobortis, tellus sit amet interdum maximus, justo elit faucibus turpis, quis volutpat magna lorem id purus. Donec urna erat, semper non egestas eget, sodales at quam. Etiam interdum tincidunt purus, eu tempor mauris consequat consequat. Aenean feugiat sapien enim, et cursus sem ultrices vitae. Vivamus enim purus, malesuada sit amet turpis sed, porta consequat tellus. Orci varius natoque penatibus et magnis dis parturient montes, nascetur ridiculus mus. Aenean vitae nisl convallis, venenatis magna et, aliquet sem. Proin diam est, rutrum at auctor vel, gravida nec magna. Aliquam erat volutpat. Nunc eu tempus ex. Praesent eget aliquam metus, ut faucibus sapien. Donec porttitor, dolor et accumsan volutpat, est neque sagittis libero, eu lobortis lacus lorem sed diam.\\


% \pnasbreak splits and balances the columns before the references.
% If you see unexpected formatting errors, try commenting out this line
% as it can run into problems with floats and footnotes on the final page.
\pnasbreak

\bibliography{paper}

\end{document}
