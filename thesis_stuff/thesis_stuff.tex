\documentclass[10pt]{article} % Font size - 10pt, 11pt or 12pt

\usepackage[hmargin=1.25cm, vmargin=1.5cm]{geometry} % Document margins

\usepackage{marvosym} % Required for symbols in the colored box

\usepackage[usenames,dvipsnames]{xcolor} % Allows the definition of hex colors

% Fonts and tweaks for XeLaTeX
\usepackage{fontspec,xltxtra,xunicode}
\defaultfontfeatures{Mapping=tex-text}
%\setmonofont[Scale=MatchLowercase]{Andale Mono}

% Colors for links, text and headings
\usepackage{hyperref}
\definecolor{linkcolor}{HTML}{506266} % Blue-gray color for links
\definecolor{shade}{HTML}{F5DD9D} % Peach color for the contact information box
\definecolor{text1}{HTML}{2b2b2b} % Main document font color, off-black
\definecolor{headings}{HTML}{701112} % Dark red color for headings
% Other color palettes: shade=B9D7D9 and linkcolor=A40000; shade=D4D7FE and linkcolor=FF0080

\hypersetup{colorlinks,breaklinks, urlcolor=linkcolor, linkcolor=linkcolor} % Set up links and colors

\usepackage{fancyhdr}
\usepackage{amsmath}
\usepackage{physics}
\usepackage{amssymb}
\pagestyle{fancy}
\fancyhf{}
% Headers and footers can be added with the \lhead{} \rhead{} \lfoot{} \rfoot{} commands
% Example footer:
%\rfoot{\color{headings} {\sffamily Last update: \today}. Typeset with Xe\LaTeX}

\renewcommand{\headrulewidth}{0pt} % Get rid of the default rule in the header

\usepackage{titlesec} % Allows creating custom \section's

% Format of the section titles
\titleformat{\section}{\color{headings}
\scshape\Large\raggedright}{}{0em}{}[\color{black}\titlerule]

\title{Dynein project log}
\author{Elliott Capek}
\titlespacing{\section}{0pt}{0pt}{5pt} % Spacing around titles

\begin{document}

\maketitle{}

\section{Equipartition Theorem}
The Equipartition Theorem (ET) predicts the average energy per degree of freedom of a system with energy which depends on the square of each degree of freedom. Mathematically is is expressed as:

\begin{align*}
  Z &= \int_{-\infty}^{\infty} e^{-\beta Ax^2}dx = \sqrt{\frac{\pi}{\beta A}}\\
  <E> &= - \frac{d\ln Z}{d\beta} = -\left(\sqrt{\frac{\beta A}{\pi}}\right)\left(\frac{\sqrt{\beta A}}{2\sqrt{\pi}}\right)\frac{-\pi}{A\beta^2}\\
  &= \frac{1}{2}k_BT\\
\end{align*}

where Z is the partition function and x a degree of freedom.\\

For our model, each domain feels three separate forces:

\textbf{Brownian forces}: Random forces due to water particle collisions. Gaussian-sampled with a
variance $\sqrt{\frac{\gamma k_BT}{dt}}$. Brownian forces are somehow necessary for the ET energy
to be realized.\\

\textbf{Conformational forces}: Spring forces from the molecule deviating from its lowest-energy
conformation. The energy from these forces is quadratic with respect to angle, and so each free
angle should contribute to the average energy by ET.\\

\textbf{Internal forces}. Tension forces (constraint forces) felt between domains which keep
the protein rigid. These are not quadratic, so it is unclear how they contribute to energy!\\

\subsection{Onebound case}
The conformational forces are:\\
\begin{align*}
  |\vec{F_{ub}}| &\approx \frac{c_m\theta_{um}}{L_s}\\
  |\vec{F_{um}}| &\approx \frac{c_m\theta_{um}}{L_s} + \frac{c_m\theta_{um}}{L_t} + \frac{c_t\theta_{t}}{L_t}\\
  |\vec{F_{t}}| &\approx \frac{c_m\theta_{um}}{L_t} + 2\frac{c_t\theta_t}{L_t} + \frac{c_m\theta_{bm}}{L_t}\\
  |\vec{F_{bm}}| &\approx \frac{c_t\theta_t}{L_t} + \frac{c_m\theta_{bm}}{L_t} + \frac{c_m\theta_{bm}}{L_s} + \frac{c_b\theta_{bb}}{L_s}\\
  |\vec{F_{bb}}| &\approx \frac{c_m\theta_{bm}}{L_s} + \frac{c_b\theta_{bb}}{L_s}\\
\end{align*}

From this we can come up with rough order-of-magnitude estimates of the energy of each domain is:

\begin{align*}
  PE_{ub} &\approx \frac{c_m}{L_s}\theta_{um}^2 + \frac{c_m}{L_t}\theta_{um}^2\\
  PE_{t} &\approx \frac{c_t}{L_t}\theta_{t}^2\\
  PE_{bm} &\approx \frac{c_m}{L_s}\theta_{bm}^2 + \frac{c_m}{L_t}\theta_{bm}^2\\
  PE_{bb} &\approx \frac{c_m}{L_s}\theta_{um}^2\\
\end{align*}

From this we can see that the conformational PE of each domain depends on different combinations of
stalk/tail length and spring constants.

\textbf{When is equipartition behaved?}
It seems like ET should be behaved when we have an even balance of Brownian and conformational forces:

\begin{align*}
  F_{conf} &\approx F_{Brownian}\\
  \frac{c}{L}\theta &\approx \sqrt{\frac{k_BT\gamma}{dt}}\\
\end{align*}

Is this reflected in our simulation? Well take a look at %% Fig ($\ref{fig:latent-heat}$)
...

%% \begin{figure}[h!]
%%   \centering
%%   \includegraphics[width=0.5\textwidth]{../figures/ob_equipartition_vs_force_ratio.pdf}
%%   \caption{Onebound conformational energy follows equipartition best when the ratio of conformational
%%   forces to Brownian forces is one.}
%%   \label{fig:equipartition_vs_force_ratio}
%% \end{figure}

\end{document}
