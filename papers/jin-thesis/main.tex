\documentclass[12pt]{report}
\usepackage[utf8]{inputenc}
\usepackage{graphicx}
\usepackage{geometry}
\geometry{margin=1in}
\usepackage{fancyhdr}
\pagestyle{fancy}

\usepackage{float}
\usepackage{amssymb}
\usepackage{amsmath}
\usepackage{amsthm}
\usepackage{ulem} 

\usepackage{caption}
\usepackage{subcaption}

\usepackage{titlesec}
\usepackage{setspace}
\usepackage{hyperref}
\usepackage[labelfont=bf, textfont=md]{caption}

\usepackage[square,numbers]{natbib}
\usepackage{indentfirst}

\floatstyle{plaintop}
\newfloat{algorithm}{thb}{lop}
\floatname{algorithm}{Algorithm}
\newenvironment{alg}
{\hrulefill\begin{enumerate}}
{\end{enumerate}\hrulefill}

\title{
	\Huge{Modeling the mechanisms of dynein's step using Monte Carlo simulations and Brownian dynamics}\vspace{1em}\\
	\Large{Jin Kiatvongcharoen} \vspace{0.5em}\\ 
	\normalsize{Department of Physics, Oregon State University}\vspace{2em}\\
	\includegraphics[width=0.4\columnwidth]{OSU_logo.png}\vspace{1em}\\
	\normalsize{An undergraduate thesis advised by Dr. David Roundy}\\
	\normalsize{in partial fulfillment of the requirements for the degree of}\\
	\normalsize{Baccalaureate of Science in Physics}\vspace{1.5 em}\\
	\normalsize{Submitted \today}\\


}
\date{}

\begin{document}
    \maketitle
    \onehalfspacing


	\chapter*{Abstract}
	Molecular motor proteins are responsible for the cell’s ability to move, divide, and spatially organize itself. A defect within any of the three motor proteins can cause severe damage to the cell’s functionality and lead to neurological diseases. In order to fully understand the cause of such possible diseases, the motor protein dynein and more specifically its structure has been extensively studied. However, there has been limited research focused on the motion of dynein. To investigate this, we will simulate dynein’s stepping mechanisms using Monte Carlo methods and Brownian dynamics. The dynein will be computationally replicated with a particle-rod model undergoing Brownian motion in a fluid, where its stepping will be repeatedly simulated to generate an ensemble of statistics, via the Monte Carlo. The program will optimize dynein’s stepping patterns by fitting parameters according to experimental results of dynein’s step length, time, and probability of stepping. 
	
    \begin{spacing}{1.2}
	\tableofcontents
	\listoffigures
	\end{spacing}
	
	
	\chapter{Introduction}
	\textit{Introduce topic by talking about cells. Attention grabber, relate biophysics study to cancer reserach and many areas where we need to focus our research. Short and sweet, maybe 1 or 2 short paragraphs.}

	Neuroegenerative diseases, such as Alzheimer's and Huntinton's disease, have consistently been one of the leading causes of death in the United States. Scientists all over the world struggle to find the leading causes of these diseases and continue to hope for a breakthrough in science in attempt to prevent or even cure these fatal diseases. Although this breakthrough has yet to come, researchers discovered that one of the central causes of cell degeneration is motor protein failure. More specifically, defects in the motor protein dynein has been linked to neurodegeneration due to the cell's funtional dependence on dynein and its mechanical properties. However, unlike dynein's motor protein siblings, dynein possesses a unique structure causing it to stochasitcally move around the cell and unpredicatable in nature. Thus, fully understandding the mechanisms of dynein can allow us to further conclude causes of its failure, putting us one step closer to solving the breakthrough. 

\section{Background}

\begin{figure}[H]
	\centering
	\includegraphics[width=0.6\columnwidth]{Figures/dynein_walking_art.jpg}
	\caption[Artist Rendition of Dynein]{\textbf{Artist Rendition of Dynein} \cite{JohnsonArt}}
	\label{fig:final_disp}
\end{figure}

Cells are complicated. Despite their degeneracies, cells have the ability to move, perfectly divide, and spatially organize themselves. They are constructed by organelles surrdounded by a network of protein filaments called the cytoskeleton. The cytoskeleton is composed of microtubules (MT) that act as a walkway for motor protein soldiers to carry out the functions of the cell. These motor proteins consists of kinesin, myosin, and dynein. 


\textit{THIS IS STRAIGHT FROM PROPOSAL, NEED TO FIX THIS AND ADJUST SO THAT MORE DYNEIN DETAILS GOES INTO CHAPTER 2}

In a eukaryotic cell, dynein is one of the three motor proteins that are responsible for the cell’s ability to move, divide, and spatially organize itself. Similar to the more researched kinesin, dynein conveys cargo along the microtubule track by using ATP to power its two binding domains (its “feet”). However, its walking-like movements along the track are very unpredictable and can vary in terms of distance and direction. Dynein’s two feet can also act independently from each other causing much more erratic and stochastic steps. However, despite dynein’s stochastic nature, dynein is still able to achieve processive motion due to the functions of its structure. Structurally, dynein is composed of two motor heavy chain subunits of linked amino acids, called polypeptides, each which are separated into domains. These domains are the tail domain, the linker domain, the six AAA+ domains, and the microtubule binding domains (See Figure 1). The AAA+ domains are responsible for ATP hydrolysis, in which converts chemical energy stored in ATP into mechanical work causing dynein’s motility. 

Because dynein’s stepping is unpredictable in nature, its stepping mechanism has not been intensely studied as compared to its structure. Questions regarding the electrostatic interactions with the microtubule, ring stacking, discrete microtubule binding sites, or elasticity of the stalk are yet to be answered. A known and supported model to describe dynein’s motion is the powerstroke model, where ATP binding to an AAA+ domain triggers conformational changes that lowers the affinity of the binding domain for the microtubule, causing it to unbind and take a step . While the powerstroke model and other theorized stepping models exists, there are no such studies which use molecular dynamics to verify if these theorized mechanisms are feasible for the dynein motor to produce. There are studies that incorporate a computational model of dynein, but many use chemical rate transitions and assume independence of steps without simulating the precise protein dynamics within its step.


\section{Motivation}
\textit{Briefly write about the importance of dynein, how dynein is researched when studying neurological diseases, dynein stepping is understudied compared to other motor proteins, hard to model dynein because of random walking, there are almost no computational models of dynein that uses molecular dynamics, modelling dyenin can give us better understanding of why it moves the way it does, specific motivation for model: new research found that there is interhead coordination when stepping so maybe we can use that to base our model off of.}
\par

To bridge this gap, we propose a coarse-grained model of dynein that assumes a particle-rod structure for its various domains and uses Brownian motion to simulate the model’s behavior in physically realistic drag and diffusion conditions, while efficiently collecting a wide ensemble of statistics using Monte Carlo methods. We chose to use Brownian inspired motion to simulate dynein’s molecular dynamics in order to replicate dynein’s stochastic behavior under realistic conditions and produce its unpredictable stepping patterns observed from experiment. We call this process Brownian dynamics, as it replaces interactions the domains have with solvent molecules with a stochastic force and allows us to simulate large time scales compared to other molecular dynamics simulation. Likewise, Monte Carlo methods will allow us to visualize dynein as a system and associate its possible configurations as states. With our simulation following these methods, we can generate an ensemble of statistics and compare measured quantities with experiment. We intend to use this model to reproduce experimental measures and help verify existing understandings concerning the properties of dynein’s stepping mechanism. One of which being a question regarding dynein’s inter-step correlation. 
	\chapter{Theory}
	\section{Model}
To reliably reproduce experimental observations of dynein's step, a model of dynein should responsibly conserve dynein's spatial information and complex structure without limiting the physical interactions with its environment. Doing so computationally introduces a strict balance between model accuracy and computational efficiency. In an attempt to best satisfy this difficult balance, we propose a simple model of dynein under a series of assumptions that benefits quick simulations of dynein taking a step. 


\subsection{2D Rigid Rod Model}
Since our goal is to investigate dynein's interhead coordination during its forward directed stepping, we decided to model dynein as a two-dimensional system of circular domains held together by massless rigid rods. These cirular domains embody the two binding domains, two motor domains, and one tail domain, where each domain act as independent angular springs with their own spring constants. A labeled picture of the model is shown below in Figure (\ref{fig:model}). 

\begin{figure}[H]
	\centering
	\includegraphics[width=0.6\columnwidth]{Figures/model-cartoon.png}
	\caption[Dynein Model]{\textbf{Dynein Model.} A model of the dynein in the both-bound state where both binding domains are bounded to the microtubule. The springs are only a visualization of how they control the domains and are not a part of the model geometry. \cite{Capek2017}}
	\label{fig:model}
\end{figure}

Despite dynein's step varying in both length and direction, we limit our model in one direction to eliminate the off-axis and only focus on dynein's forward stepping. This would simplify the asymmetry of the microtubule when dynein is stepping on the $\alpha$ or $\beta$ tubulin. We also assume massless stalks and linkers due to the strong drag force terms dominating the mass of the rods and domains in the equations of motion. The reasoning for this will be later discussed in Section \ref{sec:BrownianDynamics} when we introduce Brownian dynamics. Although the stalk and linker of real dynein definitely have mass, we counteract the loss of interactions the rods experience with the environment by increasing the radii of the domains abitrarily 10\% larger than experimental measurements.

\subsection{Domains as Angular Springs}
To best simulate the stretching and forward kicking motion of the power stroke, we define the domains to act as springs with spring constants $c_b$, $c_m$, and $c_t$. This is analogous to how humans walk, where the domains correspond to flexible joints when taking a step. Defining the domains as springs influences forward bias walking by introducing a restoring force in each domain using Hooke's law, 
\begin{equation}
    F_i=-c_i(\theta_i-\theta_{i,eq}),
\end{equation}
where $c_i$ is the spring constant, $\theta_i$ is angle corresponding to the $i^{th}$ domain, and $\theta_{i,eq}$ is the equilibrium angle in which biases the motion. Using this, we can calculate the energy of each domain by integrating and arriving to the familiar equation for spring energy:
\begin{equation} \label{eqn:energy}
    U_i=\frac{1}{2}c_i(\theta_i-\theta_{i,eq})^2.
\end{equation}
%Since dynein is symmetric for each heavy chain leg, the equilibrium angle will be identical for both binding and motor domain pairs, i.e. $\theta_{b,eq}$ is defined as the equilibrium angle for both binding domains, while $\theta_{m,eq}$ is defined as the equilibrium angle for both motor domains. 
This equation is used throughout the simulation to determine the total energy of our dynein at a given conformation. 


\subsection{Two-State System}
According to the mechanochemical cycle (Figure \ref{fig:MechanochemicalCycle}), dynein can be in eight states over the course of a single step. However, many of these states possess similar conformations, where the dynein either has low affinity (unbinded domain diffusing above microtubule) or high affinity. Since we assume simple ATP exchange and that the two physical states are during binding and unbinding, we categorize the cycle into two main conformations: the pre-stroke one-bound state and the post-stroke both-bound state.

\begin{figure}[H]
	\centering
	\includegraphics[width=0.9\columnwidth]{Figures/OB_vs_BB.PNG}
	\caption[One-Bound vs. Both-Bound]{\textbf{One-Bound vs. Both-Bound} \textit{Left}: Prestroke one-bound state with labeling scheme 'b-’ and ‘u-’ for bound leg and unbound leg. \textit{Right}: Poststroke both-bound state with labeling scheme ‘n-’ and ‘f-’ for ``near" leg and ``far'' leg. \cite{Capek2017}.}
	\label{fig:OBvsBB}
\end{figure}

Differentiating the two conveniently biases the conformational changes by defining different equilibrium angles for both states. Since dynein is symmetric for each heavy chain leg, the equilibrium angles for the both-bound state is identical for both binding and motor domain pairs, i.e. $\theta_{b,eq}$ is defined as the equilibrium angle for both binding domains, while $\theta_{m,eq}$ is defined as the equilibrium angle for both motor domains. However, in the one-bound state, the unbound leg exhibits the powerstroke and kicks out. Therefore, the equilibrium angles for the motor domains differ in order to bias the unbound leg to straighten and perform the powerstroke. This straightening mechanic resembles the linker being primed before allowing the binding domain to rebind.


\subsection{Transitioning Between States}
%\textit{Physics about conformational energy changes. How we model transitioning between BB state and OB state using transitioning rates and spring energies.}

Similar to other simulations, the binding and unbinding process is governed by rate-limiting steps, where each process is associated with binding rates. These binding rates are model parameters defined by binding constants with units of probability per time. When the dynein is in the one-bound state, the binding domain depends on two rebinding parameters: the affinity transition time ($t_a$) and the binding rate ($\rho_b$). We assume that the one-bound can only bind when high affinity completely transitions into low affinity and when the domain is within a vertical range of the microtubule, i.e.
\begin{equation}
	t>t_a=\frac{1}{k_a}
\end{equation}
and
\begin{equation}
	\rho_b=k_b(1-H(r_{ub,y}-\epsilon)).
\end{equation} 
Here, $t$ is the total one-bound time, $k_a$ is the affinity transition rate, $k_b$ is the binding rate constant, $H$ is the Heaviside step function, $r_{ub,y}$ is the y position of the unbound binding domain, and $\epsilon$ is the range within the microtubule.

When the dynein is in the both-bound state and wants to unbind, we incorporate a stepping bias in the unbinding rate. As supported from Yildiz in Figure (\ref{fig:trailingbias}), the likelihood of the trailing leg unbinding is greater than the leading leg when the distance between the two legs increases. 


\begin{figure}[H]
	\centering
	\includegraphics[width=0.6\columnwidth]{Figures/trailingbias.png}
	\caption[Probability of Unbinding Bias]{\textbf{Probability of Unbinding Bias.} The probability of either trailing or leading steps which correspond to the model's ``near" or ``far" leg unbinding, respectively. \cite{Dewitt2012}.}
	\label{fig:trailingbias}
\end{figure}

To favor trailing steps, we define an angle dependent exponential factor within the binding domains and mathematically express this as,
\begin{equation}
	\rho_{ub}=k_{ub}e^{C(\theta_b-\theta_{b,eq})}.
\end{equation}
We introduce a new model parameter $C$, the exponential unbinding constant, that controls this bias. To achieve trailing step bias, we define $C$ to be negative because the trailing $\theta_b$ tends to be less than the leading $\theta_b$ and  when $\theta_b$ is less than $\theta_{b,eq}$, the unbinding rate becomes large.


\section{Simulation}

The key difference between the works in \cite{Capek2017, waczak2019drunken} and this paper is the implementation of a Monte Carlo algorithm in replacement for the Brownian dynamics both-bound state. In addition to the large both-bound time mentioned earlier, they noticed that the initial both-bound configuration of the motor and tail domains do not impact the stepping, giving a better incentive to apply a Monte Carlo approach. This new method of analyzing the both-bound state will significantly improve computational efficiency while still maintaining the dynamical simulation of the actual step in the one-bound state. 

%These quick simulations are governed by a Monte Carlo algorithm, while the movement of the model is dictated by equations of motion describing domains interacting with a force of drag and a random force. This type of molecular simulations are more commonly known as Brownian dynamics. (\textbf{Possible take out sentence introducing MC and Brownian dynamics for later})


\subsection{Brownian Dynamics for One-Bound}
\label{sec:BrownianDynamics}

When simulating dynein's one-bound motion, the domains of constant mass must obey classical Newtonian physics and follow Newton's laws of motion. That is,
\begin{equation}
	\sum_{i}\textbf{F}_i=m\ddot{\textbf{r}}.
\end{equation} 
For the domains in the aqueous environment of the cell, linear drag force is
\begin{equation}
	\textbf{F}_{drag}=-\gamma \dot{\textbf{r}},
\end{equation}
where $\gamma$ is the drag coefficient defined by Stokes' law,
\begin{equation}
	\gamma=6\pi\eta a.
\end{equation}
This equation specifies the drag term by a given radius $a$ and liquid viscocity $\eta$. 

A second force is the random force inspiring the Brownian motion. \textbf{F}$_{rand}$ describes the random interactions between the domains and the solvent paticles. \textbf{F}$_{rand}$ is sampled from a zero-centered Gaussian with variance
\begin{equation}
	\sigma^2=\sqrt{\frac{2\gamma}{\beta\Delta t}},
\end{equation}
where $\beta=\frac{1}{k_bT}$  and $\Delta t$ is a time-step. 

The final force is the tension force, \textbf{F}$_T$, between the domains \textit{via} their connecting rods. Incorporting all these forces, we produce the \textit{Langevin Equation}:
\begin{equation}
	-\gamma \dot{\textbf{r}} + \textbf{F}_{rand} + \textbf{F}_T = m\ddot{\textbf{r}}.
\end{equation}
To properly demonstrate Brownian dynamics, we must assume strong drag where $|\gamma \dot{\textbf{r}}|\ll |m\ddot{\textbf{r}}|$. In other words, we assume that the immense viscosity in the environment restricts the ability for the domains to accelerate. Simplifying the \textit{Langevin Equation} and solving for $\dot{\textbf{r}}$, we get:
\begin{equation}
	\dot{\textbf{r}}=\frac{1}{\gamma}(\textbf{F}_{rand} + \textbf{F}_T)
\end{equation}
which is a differential equation we can solve for the equations of motion for the domains in the one-bound state. The derivation and final equations of motion can be found in \cite{Capek2017}


\subsection{Monte Carlo for Both-Bound}
Since dynein spends most of its stepping time in the both-bound state, we safely assume dynein equilibriates before unbinding. This allows us to use Monte Carlo methods for analyzing dynein's step by interpreting our model as a system. Typically, Monte Carlo methods analyze equilibrium theromodynamics of fluids, where a system of atoms possess many different microstates \cite{lim2007vorticity}. For this model, dynein is the system and the many possible both-bound configurations are the ``microstates." However, unlike the more commonly known Markov Chain Monte Carlo, each both-bound configuration is independent and can generate different total energy values. This produces a large ensemble of of both-bound configurations, which we sample to create stepping statistics. Eventually, these statistics will reduce to probability distributions for important variables.

With the assumption of thermodyamic equilibrium during the both-bound state, we use a Boltzmann distribution to define the probability of a configuration and calculate the relative unbinding probability with a proportionality expression. That is, 
\begin{equation}
	P_{ub}\propto e^{-\beta E_{total}}.
\end{equation}
With this Boltzmann factor, we can also generate a partition function, 
\begin{equation} \label{eqn:partition}
	Z=\sum_{i}e^{-\beta E_{total,i}},
\end{equation}
where $i$ refers to the specific both-bound configuration. In order to guarantee an accurate thermodynamic ensemble of both-bound configurations, the simulation must run a large number of trials to ensure that the sample space is covered well. The thermodynamic ensemble then generates an ensemble of steps that can eventually produce smooth statictics. With the partition function defined in Equation (\ref{eqn:partition}), we also define ensemble averages for both-bound statistics, such as the average unbinding rates, unbinding probability, and both-bound time. The equations to do so are all shown below in that order:
\begin{equation}
	\langle\rho_{\textit{ub, trail.}}\rangle=\frac{\sum_{i}\rho_{\textit{ub, trail.}} e^{-\beta E_{total,i}}}{Z}, \indent \langle\rho_{\textit{ub, lead.}}\rangle=\frac{\sum_{i}\rho_{\textit{ub, lead.}} e^{-\beta E_{total,i}}}{Z}
\end{equation}
\begin{equation} \label{eqn:ProbTrail}
	\langle P_{\textit{ub, trail.}}(L)\rangle = \frac{\langle\rho_{\textit{ub, trail.}}\rangle}{\langle\rho_{\textit{ub, trail.}}\rangle + \langle\rho_{\textit{ub, lead.}}\rangle}
\end{equation}
\begin{equation}
	\langle t_{bb}(L) \rangle =\frac{1}{\langle\rho_{\textit{ub, trail.}}\rangle + \langle\rho_{\textit{ub, lead.}}\rangle},
\end{equation}
where $trail.$ and $lead.$ signifies a trailing or leading step.



%\textit{STRAIGHT FROM PROPOSAL. FIXME}

%Our entire model of dynein will be coded with a combination of both Python and C++. The structural features of dynein will be captured with a two-dimensional geometric model of circular domains connected by rigid rods. The dynamics of dynein will be captured by imposing both equilibrium and Brownian forces on each circular domain of the model, while the chemical properties will be modeled by sporadically transitioning between two states: a poststroke both bound state, and a prestroke one bound state. These two states are shown below in Figure 2.
%
%The simulation for stepping will consist of running multiple independent steps that starts in a both bound configuration and transitions into the Brownian dynamics one bound state. In this state, each domain will undergo Brownian forces until the unbounded leg diffuses back onto the microtubule, thus completing a step. The Brownian dynamics only affects the one bound state since that is the only time (in our model) the domains undergo molecular interaction with its surroundings to achieve processive motion. This Monte Carlo method of collecting an ensemble of statistics after many simulations will lead to probability distributions of measured variables that can be compared with experimental results. The common experimental results analyzed in this field are the distribution of step lengths (defined as the final displacement of the “stepping” leg minus the initial displacement), stepping times, probability of the binding domain unbinding, and the velocity of a step. Once sufficient data is collected for each listed variable, plotting scripts on Python will be used to visualize the possible relationships between these variables and generate analysis for our model’s stepping patterns. Currently, a paper published by Dr. Ahmet Yildiz from the University of Berkeley proposed a possible relationship within dynein’s stepping, implying inter-step correlation [3]. Our model aims to reproduce these results by finding a set of parameters that matches Yildiz’s observation in the form of a linear regression of the step lengths against the initial binding domain separation. We hope to either support Yildiz’s claim or infer some new property of dynein that generated Yildiz’s observation.
	\chapter{Methods}
	%\textit{All code for simulation can be accessed from GitHub through this link:}

The entire model and simulation code can be found here on \url{https://github.com/kiatvonj/dynein_walk}, where the model and simulation were coded using both Python and C++. The both bound Monte Carlo was coded on Python due to the calculations being less computationally rigorous and time independent, while the one bound Brownian dynamics was coded on C++ to utilize the faster computation speed for the time  dependent Brownian motion.

\section{Simulation}
%\textit{Picking Random angles to create random distribution of dynein configurations. Make sure to clarify that the simulation is split into two parts, the monte carlo both bound and brownian one bound. It is done this way in order for MC to generate ensemble of data rather quickly.}\\
%\textit{Include a flow chart.}
Since we want to generate an ensemble of steps, we use the Monte Carlo method to simulate a single step at a time, as opposed to a walk. This allows us to simplify the code into two respective parts (the Monte Carlo both bound and the Brownian dynamics one bound) that cycles repeatedly after the model takes a step. The simulation starts in the both bound state, where the Monte Carlo randomly picks a both bound configuration and calculate the total energy. If the configuration successfully unbinds, then we use the positions of the domains as initial conditions for the one bound simulation and execute the Brownian dynamics. Once the one bound dynein diffuses onto the microtubule and rebinds, the simulation repeats the cycle and generates a completely different both bound configuration for another step. However, the big caveat with this method is that we must initialize the distance between the binding domains, $L$, in order to ensure that our ensemble covers the whole sample space of possible both bound configurations. This will force the measured probabilities to be a function of $L$ and will be addressed later in Section (\ref{sec:DataAna}). The algorithm for the both bound process is shown below:

\begin{algorithm}[H]
	\caption{Monte Carlo Both Bound}
	\label{alg:MonteCarlo}

	\begin{alg}
	\item Initialize an unsigned distance, $L$, between the binding domains.
	
	\item Randomly pick 4 angles ($\theta_0$, $\theta_1$, $\theta_2$, $\theta_3$) to generate a completely random configuration of dynein in space. (\textit{Include picture})
	
	\item Calculate the distance between the binding domains and check if the distance is within a range of $L\pm l$, where $l$ is arbitrarily small.
	
	\item Rotate the configuration so that both binding domains are on the microtubule and recalculate angles with the naming convention in Figure (\ref{fig:OBvsBB}).
	
	\item Calculate the total energy of the configuration by summing the energy of each domain (using Equation (\ref{eqn:energy})),
	\begin{equation}
		E_{total, i}=\sum_{k}\frac{1}{2}c_k(\theta_k-\theta_{k,eq})^2.
	\end{equation}
	Here, $i$ refers to the specific both bound configuration and $k$ is the domain.	
	
	\item Calculate the relative unbinding probability from a Boltzmann distribution and add a term to the partition function:
	\begin{equation}
		P_{ub} \propto \rho_{ub}e^{-\beta E_{total, n}}\kappa,
	\end{equation}
	\begin{equation}
		Z=\sum_{n}e^{\beta E_{total, n}}.
	\end{equation}
	We defined a $\kappa$ to normalize the relative unbinding probability and keep it less than 1. 
	
	\item ``Roll a die" for a value within the range of $[0,1]$ and check if the value is less than $P_{ub}$. If not, repeat algorithm from step 2.
	
	\item If dynein unbinds, initialize the one bound state with the both bound configuration, run the Brownian dynamics simulation of the step, and wait until it rebinds.
	
	\item Collect statistics about final displacement and one bound time. Repeat algorithm from step 2. 
	
	\end{alg}

\end{algorithm}

A more in-depth explanation and rigorous flow chart for the one bound Brownian dynamics simulation can be found here \cite{Capek2017, }.

%\section{Time Evolution}
%\textit{Simulate over a delta t during one bound and every iteration check for rebinding. How are the probabilities of binding and unbinding affected by the time.}

\section{Constants}
\textit{How we defined our constants based on experiment and pictures.  Need to include figures here of experiment dynein and how we measure the pre and post stroke equilibrium angles.}

\begin{figure}[hbt!]
	\centering
	\includegraphics[width=0.3\columnwidth]{../../plots/burgess-model-figure.pdf}
	\includegraphics[width=0.5\columnwidth]{../../plots/grotjahn-model-figure.pdf}%
	\caption[Experimental Pictures of Dynein for Model Parameters]{\textbf{Experimental Pictures of Dynein for Model Parameters} Left: the both bound model is superimposed over dynein micrograph. The arrow indicates 15nm. Image taken from \cite{burgess2003dynein}.  Right: cryo-electron tomograph of dynein with the one-bound model overlaid. Image taken from \cite{grotjahn}.} 
	\label{fig:ModelParams}
\end{figure}


\section{Parameter Fitting}
\textit{How we will fit our parameters to agree with Yildiz' experimental data of dynein stepping.}

\subsection{Rate Constants}
\textit{Fitting our rate constants were a huge part of our model and how we made sure our dynein can match experimentalist results.}

\section{Model Validation}
\textit{Any tests of our model to make sure the code is reasonably sound and that the physics makes sense. Make sure there is no bugs in code. What was the bug testing procedure.}

\section{Data Analysis} \label{sec:DataAna}
\begin{equation}
	p(x_i,x_f)=p(x_f|x_i)p(x_i)
\end{equation}
Bayes' Theorem
\begin{equation}
	p(x_i|L_i)=\frac{p(L_i|x_i)p(x_i)}{p(L_i)}
\end{equation}
\[
	p(x_i|L_i)=\frac{p(x_i)}{p(L_i)} 
\]\[
	p(x_i)=p(x_i|L_i)p(L_i)
\]
Detailed Balance Equation:
\begin{equation}
	p^n(L_i)=T^n(L_f|L_i)p^0(L_i)
\end{equation}
\begin{align}
	T(L_f|L_i)&=I(L_f|x_i)p(x_f|x_i)p(x_i|L_i)\\
	&=p(L_f|x_i)p(x_i|L_i)\\
	&=p(L_f|L_i)
\end{align}




	\chapter{Results \& Discussion}
	\section{Optimized Parameters}\label{sec:Params}
The set of parameters that best fit Yildiz's data are shown in Table (\ref{tab:params}) below.

\begin{table}[H]
  \centering
  \begin{tabular}{|l | r | r | r|}
  	\hline
    Param & Model & Experimental & Source \\
    \hline
    $c_b$ & $\cb \Delta G_{ATP}$ &  & \\
    $c_m$ & $\cm \Delta G_{ATP}$ &  & \\
    $c_t$ & $\ct \Delta G_{ATP}$ &  & \\
    $k_p$ & $\kstk s^{-1}$&  & \\
    $k_b$ & $\kb s^{-1}$&  & \\
    $k_{ub}$ & $\kub s^{-1}$ & & \\
    $C$ & $\cexp$ & & \\
    $L_s$ & $\ls nm$ & $21nm$ & \cite{Burgess2003, 3vkh-cite, carter-paper}\\
    $L_t$ & $\lt nm$ & $23nm$ & \cite{Burgess2003, 3vkh-cite, carter-paper}\\
    $\theta_b$ & $\eqb$ &  120 & \cite{leschziner} \\
    $\theta_m^{\mbox{pre}}$ & $\eqmpre$ &  197 & \cite{Burgess2003}\\
    $\theta_m^{\mbox{post}}$ & $\eqmpost$ & 242 & \cite{Burgess2003}\\
    $\theta_t$ & $\eqt$ &  & \\
    $R_t$ & $\radiust$ & $8nm$ & \cite{Burgess2003}\\
    $R_m$ & $\radiusm$ & $11nm$ & \cite{Burgess2003}\\
    $R_b$ & $\radiusb$ & $3.5nm$ & \cite{Burgess2003}\\
    \hline
  \end{tabular}
  \caption{Optimized model parameters for following data sets. }
%  \caption{Parameters used in simulation. $c_b$, $c_m$ and $c_t$ are the MTBD, motor and tail spring constants, respectively. $k_b$ and $k_{ub}$ are the rates of pre-stroke to post-stroke and post-stroke to pre-stroke transitions, respectively. $L_s$ and $L_t$ are the lengths of the stalk and tail interdomain linkers. $c$ is the tension-gating factor. The $\theta$ values are the equilibrium values for each angle, where $\theta_m^{\mbox{pre}}$ is the equilibrium angle for the unbound motor in pre-stroke, whereas $\theta_m^{\mbox{post}}$ is the equilibrium angle for the bound pre-stroke motor and both post-stroke motors. $R_b$, $R_m$ and $R_t$ are the MTBD, motor and tail radii. Parameters used for all simulations unless otherwise noted}
  \label{tab:params}
\end{table}

After many trials, this combination of parameters best demonstrated the interstep correlation observed from Yildiz's experiment, while still preserving dynein's geometric constants. \textit{FIXME: Not finalized yet!! Still a WIP, need to fix and wait for simulations to finish}


\section{Stepping Plots}
The final displacement and step lengths are plotted against experiment and shown below.

\begin{figure}[H]
	\centering
	\includegraphics[width=0.7\textwidth]{/mc_plots/u_final_disp_probability_distribution_30.0_5.50e+09_1.00e+08_0.0_1.0_1.0_120.0_197.0_242.0_-0.5}
	\includegraphics[width=0.6\textwidth]{/mc_plots/u_step_length_1d_probability_density_30.0_5.50e+09_1.00e+08_0.0_1.0_1.0_120.0_197.0_242.0_-0.5}
	\caption[Final Displacement Probability Distribution]{\textbf{Stepping Probability Distribution Plots.} \textit{Top:} Two dimensional heatmap of binding domain displacement before and after a stepping cycle. Linear regression of the probability distribution indicates a dependence between the final displacement of the step and its initial displacement. \textit{Bottom:} One dimensional probability density of step length. Step length defined as final displacement minus the initial displacement. Model compared to two experimental figures of stepping patterns from \citep{Dewitt2012}.} 
	\label{fig:DataStep}
\end{figure}
\newpage
The top plot in Figure (\ref{fig:DataStep}) displays the joint probability density of having a specific final displacement and initial displacement. The linear regression from Yildiz's experiment indicates the functionality between the two with an average final displacement of 9.1 nm, shown from the y-intercept. The bottom plot in Figure (\ref{fig:DataStep}) displays the one-dimensional probability density of step lengths, defined as final displacement minus the initial displacement. The limitation of experiment was made very obvious here due to their inability to record short step lengths. This factor heavily influenced the parameter fitting because in order to best fit the linear correlation, the model dynein had to take very fast and short steps. Since Yildiz observed a linear regression that averaged close to 0 nm step lengths without being able to record 0nm, the model dynein was forced to take steps with very quick rebinding. The step lengths close to 0 nm from Experiment Fig 3A were lateral steps in the off-axis. 

We fit the pre-exponential unbinding factor, $C$, based on experimental probability of having a trailing (lagging) step, as shown below.

\begin{figure}[H]
	\centering
	\includegraphics[width=0.7\columnwidth]{../../plots/mc_plots/prob_lagging_vs_init_L_-0.5}
	\caption[Probability of Lagging Step]{\textbf{Probability of Lagging Step}}
	\label{fig:ProbTrail}
\end{figure}

\textit{Need to include time plots and maybe more figures. Talk about the short time step. Sorry this is also still a WIP}

\section{Agree with Experiment?}
\textit{Whole point of parameter fitting was to agree with Yildiz and/or other experimental data. Did we do a good job? }

\textit{Need to eventually bring up the issue of being able to fit both the 2d hist linear regression and step length plot: If we want to fit linear regression, we must have short steps, but if we have short steps, the step length plot will not match at steps $>$ 20nm. Hard to agree with experiment when our dynein is forced to take short steps. Mention time too.}

\textit{\textbf{Note for Assignment 9.1:} I apologize that this section is still incomplete and a skeleton, but there were some hiccups in the code that delayed the simulations and may change the agreement based on how we solve the bug. I felt it was best to occupy my time trying to polish the code instead of writing a generalized version of this section.}
	\chapter{Conclusion}
	\section{Findings}
\textit{What do our results indicate? Do we think our model correctly represents real dynein and its stepping patterns?}

The model has the flexibility to replicate many sets of experimental data. The most unique set is Yildiz and we were able to reproduce his observations to the best of our ability with the set of parameters we used. We discovered an inherent problem when trying to fit to data that cannot observe 0nm steps. We usec Monte Carlo that produces a smooth probability distribution because of running large numbers of simulations and law of large numbers. However, In order to best fit the linear regression, the simulation must average 0nm step lengths, meaning we need to force parameters so that model dynein takes very quick and short steps. But, if we do this, the average step length will be a lot less than what they observe and the model will not be able to achieve large steps. 



\section{Further Work}
\textit{How can we improve our model?}

We can improve our model by implementing a third dimension that utilizes the off-axis. We could always run more simulations with different sets of parameters until we find the perfect set. We could implement a optimization process of maximizing and minimizing all parameters so that we get a large range of combinations and eventually narrow down to a parameter space that cleany replicates data. 


\newpage
\cite{Burgess2003} \cite{Cianfrocco2015mechanism} \cite{Dewitt2012} \cite{Capek2017, waczak2019drunken}
\cite{GoodsellArt} \cite{eschbach2011cytoplasmic}
\cite{rao2019molecular} \cite{manna2020mechanistic} \cite{desantis2017lis1} 
\cite{elshenawy2020lis1} \cite{ando2020small} \cite{kinoshita2018step} \cite{qiu2012dynein} \cite{muller1973dynamic} \cite{lim2007vorticity} \cite{fehr2008kinesin} \cite{trott2018mathematical}


\bibliographystyle{unsrt}
\bibliography{references}


\end{document}
